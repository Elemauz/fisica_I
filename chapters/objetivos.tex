\section{Objetivos}

\subsection{Objetivo General}
Desarrollar tres simuladores interactivos de física que permitan visualizar y analizar los principios fundamentales de la cinemática, dinámica y estática mediante interfaces gráficas intuitivas, modelado físico preciso y visualizaciones en tiempo real.

\subsection{Objetivos Específicos}

\begin{enumerate}
    \item Implementar interfaces gráficas responsivas que permitan a los usuarios modificar parámetros físicos en tiempo real y observar su influencia inmediata en los sistemas simulados.
    
    \item Modelar físicamente los principales tipos de movimiento en el simulador de cinemática: movimiento rectilíneo uniforme, movimiento rectilíneo uniformemente acelerado, caída libre, tiro parabólico y movimiento circular.
    
    \item Desarrollar un sistema de análisis de fuerzas en el simulador de dinámica que incluya las leyes de Newton, fuerzas de tensión, normales, fricción y pesos, aplicadas a sistemas de múltiples cuerpos interconectados.
    
    \item Implementar un módulo de equilibrio estático especializado en sistemas de esferas entre superficies, con cálculo automático de ángulos de equilibrio y ajuste dinámico de tensiones en cuerdas.
    
    \item Crear sistemas de visualización con gráficas interactivas que muestren variables físicas relevantes (posición, velocidad, aceleración, fuerzas) en función del tiempo o otros parámetros.
    
    \item Desarrollar animaciones en tiempo real que representen visualmente el comportamiento de los sistemas físicos, incluyendo vectores de fuerza para sistemas estáticos y dinámicos.
    
    \item Implementar diagramas de cuerpo libre interactivos que muestren todas las fuerzas actuantes sobre cada elemento del sistema, con especial énfasis en sistemas estáticos complejos con múltiples puntos de contacto.
    
    \item Validar los modelos físicos comparando los resultados de simulación con las ecuaciones teóricas correspondientes a cada área de la física, verificando condiciones de equilibrio estático y dinámico.
    
    \item Optimizar el rendimiento de las simulaciones para garantizar una experiencia fluida en diferentes dispositivos, desde computadoras de escritorio hasta dispositivos móviles.
    
    \item Crear módulos de exportación y análisis de datos que permitan a los usuarios estudiar los resultados de las simulaciones mediante gráficas profesionales y tablas de datos.
\end{enumerate}
