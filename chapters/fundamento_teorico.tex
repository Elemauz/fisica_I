\section{Fundamento Teórico}

Los simuladores desarrollados abarcan tres áreas fundamentales de la mecánica clásica: cinemática, dinámica y estática. Cada una de estas disciplinas proporciona las bases teóricas necesarias para comprender y modelar el comportamiento de sistemas físicos bajo diferentes condiciones, desde el movimiento de partículas hasta el equilibrio de cuerpos rígidos.

\subsection{Cinemática: Estudio del Movimiento}

La cinemática se enfoca en describir el movimiento de los cuerpos sin considerar las causas que lo producen. Los simuladores implementados cubren los principales tipos de movimiento:

\subsubsection{Movimiento Rectilíneo Uniforme (MRU)}
Caracterizado por una velocidad constante, se describe mediante:
\begin{equation}
x(t) = x_0 + v \cdot t
\end{equation}
donde $x_0$ es la posición inicial, $v$ la velocidad constante y $t$ el tiempo.

\subsubsection{Movimiento Rectilíneo Uniformemente Acelerado (MRUA)}
Incluye aceleración constante, gobernado por las ecuaciones:
\begin{align}
x(t) &= x_0 + v_0 t + \frac{1}{2} a t^2 \\
v(t) &= v_0 + a t \\
v^2 &= v_0^2 + 2a(x - x_0)
\end{align}

\subsubsection{Movimiento Parabólico}
Combina MRU horizontal con MRUA vertical bajo gravedad:
\begin{align}
x(t) &= v_{0x} \cdot t \\
y(t) &= v_{0y} \cdot t - \frac{1}{2} g t^2 \\
v_x(t) &= v_{0x} \\
v_y(t) &= v_{0y} - g t
\end{align}
donde $v_{0x} = v_0 \cos\theta$ y $v_{0y} = v_0 \sin\theta$.

\subsubsection{Movimiento Circular}
Describe rotaciones con velocidad angular constante o variable:
\begin{align}
\omega &= \frac{2\pi}{T} \\
v &= \omega \cdot r \\
a_c &= \omega^2 r = \frac{v^2}{r} \\
\alpha &= \frac{d\omega}{dt}
\end{align}
donde $\omega$ es velocidad angular, $T$ el período, $r$ el radio, $a_c$ la aceleración centrípeta y $\alpha$ la aceleración angular.

\subsection{Dinámica: Fuerzas y Movimiento}

La dinámica estudia las causas del movimiento mediante las leyes de Newton y conceptos energéticos.

\subsubsection{Segunda Ley de Newton}
Establece la relación fundamental entre fuerza, masa y aceleración:
\begin{equation}
\vec{F} = m \cdot \vec{a}
\end{equation}
Esta ley permite calcular la aceleración resultante de fuerzas aplicadas.

\subsubsection{Energía Cinética}
Representa la energía asociada al movimiento:
\begin{equation}
K = \frac{1}{2} m v^2
\end{equation}
El teorema trabajo-energía relaciona el trabajo neto con el cambio en energía cinética.

\subsubsection{Ley de Gravitación Universal}
Describe la fuerza atractiva entre masas:
\begin{equation}
F = G \frac{m_1 m_2}{r^2}
\end{equation}
donde $G = 6.674 \times 10^{-11} \, \text{N·m}^2/\text{kg}^2$ es la constante gravitacional, $m_1$ y $m_2$ las masas, y $r$ la distancia entre centros.

\subsection{Estática: Equilibrio de Fuerzas}

La estática analiza sistemas en reposo donde la suma de fuerzas y momentos es cero.

\subsubsection{Condiciones de Equilibrio}
Para un sistema en equilibrio estático:
\begin{align}
\sum \vec{F} &= 0 \quad \text{(Equilibrio de traslación)} \\
\sum \vec{\tau} &= 0 \quad \text{(Equilibrio de rotación)}
\end{align}

\subsubsection{Sistema de Dos Esferas entre Superficies}
El simulador de estática modela un sistema donde dos esferas interactúan entre superficies en configuración en L. Las fuerzas consideradas incluyen:

\begin{itemize}
    \item \textbf{Pesos:} $W_1 = m_1 g$, $W_2 = m_2 g$
    \item \textbf{Tensión en cuerda:} Fuerza horizontal que mantiene el equilibrio
    \item \textbf{Fuerzas normales:} $N_1$ (base-esfera1), $N_2$ (pared-esfera2), $N_{12}$ (entre esferas)
    \item \textbf{Fuerzas de fricción:} $F_1 = \mu_1 N_1$, $F_2 = \mu_2 N_2$, $F_{12} = \mu_{12} N_{12}$
\end{itemize}

La geometría del sistema define relaciones angulares donde el ángulo $\theta$ entre las esferas determina las componentes de las fuerzas de contacto. La fuerza normal entre esferas se relaciona con el peso de la esfera superior mediante:
\begin{equation}
N_{12} = \frac{W_2}{\cos\theta}
\end{equation}

Las componentes de esta fuerza son:
\begin{align}
F_{12x} &= N_{12} \sin\theta \\
F_{12y} &= N_{12} \cos\theta
\end{align}

\subsubsection{Análisis de Equilibrio por Esfera}

\textbf{Esfera 1 (inferior):}
\begin{align}
\sum F_x &= T - F_1 - F_{12x} = 0 \\
\sum F_y &= N_1 - W_1 - F_{12y} = 0
\end{align}

\textbf{Esfera 2 (superior):}
\begin{align}
\sum F_x &= N_2 - F_{12x} - F_2 = 0 \\
\sum F_y &= F_{12y} - W_2 = 0
\end{align}

\subsection{Integración Numérica y Métodos Computacionales}

Para la implementación computacional de estos modelos, se emplean métodos numéricos:

\subsubsection{Método de Euler}
Para ecuaciones diferenciales del movimiento:
\begin{align}
\vec{v}_{n+1} &= \vec{v}_n + \vec{a}_n \cdot \Delta t \\
\vec{r}_{n+1} &= \vec{r}_n + \vec{v}_n \cdot \Delta t
\end{align}

\subsubsection{Resolución de Sistemas de Ecuaciones}
En estática, se resuelven sistemas de ecuaciones lineales para determinar fuerzas desconocidas, asegurando que se cumplan las condiciones de equilibrio.

La implementación de estos fundamentos teóricos en simuladores interactivos permite visualizar conceptos abstractos, experimentar con parámetros físicos y comprender la relación entre teoría y fenómenos observables, facilitando el aprendizaje mediante la exploración y el descubrimiento.