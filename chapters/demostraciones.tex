\section{Demostraciones y Experimentos}

\subsection{Cinemática: ¿Por qué 45° proporciona el alcance máximo en tiro parabólico?}

\textbf{Demostración:}
\begin{enumerate}
    \item Seleccionar la pestaña "Tiro Parabólico" en el simulador de cinemática
    \item Configurar gravedad = 9.8 m/s², velocidad inicial = 20 m/s, altura inicial = 0 m
    \item Realizar lanzamientos con ángulos de 30°, 40°, 45°, 50° y 60°
    \item Medir y registrar el alcance horizontal en cada caso usando las gráficas de posición
\end{enumerate}

\textbf{Resultado:} El ángulo de 45° maximiza el alcance porque equilibra óptimamente las componentes horizontal y vertical de la velocidad. Matemáticamente, el alcance $R$ está dado por:
\begin{equation}
R = \frac{v_0^2 \sin(2\theta)}{g}
\end{equation}
Que alcanza su máximo cuando $\sin(2\theta) = 1$, es decir, cuando $\theta = 45°$.

\subsection{Cinemática: Independencia de movimientos horizontal y vertical}

\textbf{Demostración:}
\begin{enumerate}
    \item En tiro parabólico, configurar ángulo = 45° y velocidad inicial = 15 m/s
    \item Activar las gráficas de posición X vs tiempo y posición Y vs tiempo
    \item Observar que la gráfica X es lineal (MRU) mientras la Y es parabólica (MRUA)
    \item Verificar que el movimiento horizontal no afecta al vertical y viceversa
\end{enumerate}

\textbf{Resultado:} Las componentes horizontal y vertical del movimiento son independientes, demostrando el principio de superposición en cinemática.

\subsection{Dinámica: Verificación de la Segunda Ley de Newton}

\textbf{Demostración:}
\begin{enumerate}
    \item En el módulo de Segunda Ley de Newton, fijar masa = 2 kg
    \item Aplicar fuerzas de 10 N, 20 N, 30 N y 40 N sucesivamente
    \item Registrar la aceleración resultante en cada caso
    \item Graficar Fuerza vs Aceleración
\end{enumerate}

\textbf{Resultado:} La gráfica muestra una línea recta que pasa por el origen, verificando que $F = m \cdot a$. La pendiente de la recta corresponde a la masa del objeto.

\subsection{Dinámica: Ley de Gravitación Universal}

\textbf{Demostración:}
\begin{enumerate}
    \item Configurar dos masas: m1 = 1000 kg, m2 = 2000 kg
    \item Variar la distancia entre 1 m y 10 m en incrementos de 1 m
    \item Registrar la fuerza gravitacional calculada para cada distancia
    \item Graficar Fuerza vs Distancia y Fuerza vs 1/Distancia²
\end{enumerate}

\textbf{Resultado:} La fuerza gravitacional disminuye con el cuadrado de la distancia, verificando la ley de Newton:
\begin{equation}
F = G \frac{m_1 m_2}{r^2}
\end{equation}
La gráfica F vs 1/r² muestra una relación lineal.

\subsection{Dinámica: Conservación de la Energía Cinética}

\textbf{Demostración:}
\begin{enumerate}
    \item En el módulo de energía cinética, configurar masa = 5 kg
    \item Variar la velocidad de 0 a 20 m/s en incrementos de 5 m/s
    \item Registrar la energía cinética calculada en cada caso
    \item Graficar Energía Cinética vs Velocidad y Energía Cinética vs Velocidad²
\end{enumerate}

\textbf{Resultado:} La energía cinética es proporcional al cuadrado de la velocidad:
\begin{equation}
K = \frac{1}{2} m v^2
\end{equation}
La gráfica K vs v² muestra una relación lineal con pendiente $m/2$.

\subsection{Estática: Condiciones de Equilibrio en Sistema de Dos Esferas}

\textbf{Demostración:}
\begin{enumerate}
    \item Configurar masa esfera 1 = 10 kg, masa esfera 2 = 15 kg
    \item Observar cómo el sistema ajusta automáticamente el ángulo y la tensión
    \item Verificar en los resultados que  la suma de Fx = 0 y la suma de Fy = 0 para ambas esferas
    \item Modificar las masas y observar los nuevos valores de equilibrio
\end{enumerate}

\textbf{Resultado:} El sistema siempre encuentra una configuración donde la suma de fuerzas en X e Y es cero para cada esfera, demostrando las condiciones de equilibrio estático.

\subsection{Estática: Influencia del Rozamiento en la Estabilidad}

\textbf{Demostración:}
\begin{enumerate}
    \item Configurar masas iguales (10 kg ambas esferas)
    \item Variar el coeficiente de rozamiento esfera-base de 0.1 a 0.9
    \item Observar cómo cambia la tensión requerida en la cuerda
    \item Identificar el valor mínimo de rozamiento que mantiene el equilibrio
\end{enumerate}

\textbf{Resultado:} Mayores coeficientes de rozamiento permiten configuraciones con menor tensión en la cuerda, demostrando que la fricción es esencial para la estabilidad del sistema.

\subsection{Estática: Relación Ángulo-Masa en el Equilibrio}

\textbf{Demostración:}
\begin{enumerate}
    \item Mantener constante la masa de la esfera 1 en 10 kg
    \item Variar la masa de la esfera 2 desde 5 kg hasta 25 kg
    \item Registrar el ángulo de equilibrio para cada combinación
    \item Graficar Ángulo vs Relación de Masas (m2/m1)
\end{enumerate}

\textbf{Resultado:} El ángulo de equilibrio aumenta con la relación de masas, mostrando cómo la geometría del sistema se adapta a la distribución de pesos para mantener el equilibrio.

\subsection{Cinemática: Movimiento Circular Uniforme vs Acelerado}

\textbf{Demostración:}
\begin{enumerate}
    \item En movimiento circular, configurar radio = 5 m
    \item Probar con aceleración angular = 0 (uniforme) y = 2 rad/s² (acelerado)
    \item Comparar las gráficas de velocidad angular y lineal vs tiempo
    \item Observar la aparición de aceleración tangencial en el caso acelerado
\end{enumerate}

\textbf{Resultado:} En movimiento circular uniforme, la velocidad angular es constante y solo existe aceleración centrípeta. En movimiento acelerado, aparece aceleración tangencial que modifica la velocidad lineal.

\subsection{Validación Cruzada entre Simuladores}

\textbf{Demostración:}
\begin{enumerate}
    \item En cinemática (MRUA), calcular la aceleración de un objeto de 2 kg con F = 20 N
    \item Verificar en dinámica que la misma configuración produce idéntica aceleración
    \item Usar esta aceleración para predecir posición en cinemática y verificar coherencia
\end{enumerate}

\textbf{Resultado:} Los tres simuladores producen resultados consistentes, demostrando la coherencia de los modelos físicos implementados y la relación entre cinemática y dinámica.