\section{Procedimientos y Actividades}

La metodología para utilizar los simuladores de física (cinemática, dinámica y estática) sigue un enfoque pedagógico progresivo que combina exploración guiada, experimentación sistemática y análisis de resultados. Esta estructura permite a los usuarios desarrollar competencias científicas mientras interactúan con conceptos físicos fundamentales a través de interfaces intuitivas y responsivas.

\subsection{Preparación del Entorno de Aprendizaje}

Antes de iniciar las actividades, es esencial establecer un entorno adecuado para el aprendizaje interactivo:

\begin{enumerate}
    \item \textbf{Acceso a los Simuladores}: Abrir el navegador web y cargar la aplicación correspondiente (cinemática, dinámica o estática) desde la URL proporcionada
    \item \textbf{Verificación Técnica}: Confirmar que el navegador soporte HTML5 Canvas y JavaScript moderno, verificando que todas las gráficas y animaciones se carguen correctamente
    \item \textbf{Organización del Espacio}: En dispositivos móviles, usar orientación horizontal para mejor visualización; en computadoras, asegurar resolución mínima de 1024x768 píxeles
    \item \textbf{Preparación de Registros}: Tener disponible un cuaderno digital o físico para registrar observaciones, parámetros utilizados y resultados obtenidos
\end{enumerate}

\subsection{Metodología de Exploración Guiada}

Cada simulador incluye actividades estructuradas que guían al usuario a través de descubrimientos progresivos:

\subsubsection{Simulador de Cinemática}
\begin{enumerate}
    \item \textbf{Familiarización con Controles}: Explorar cada pestaña (MRU, MRUA, Caída Libre, Tiro Parabólico, Movimiento Circular) y entender la función de cada control deslizante
    \item \textbf{Experimento de Comparación}: Configurar el mismo tiempo en diferentes tipos de movimiento y comparar las distancias recorridas
    \item \textbf{Análisis Gráfico}: Activar todas las gráficas disponibles en tiro parabólico para observar las relaciones entre posición, velocidad y tiempo
    \item \textbf{Animaciones Interactivas}: Usar los botones de animación para visualizar el movimiento en tiempo real mientras se monitorean los valores numéricos
\end{enumerate}

\subsubsection{Simulador de Dinámica}
\begin{enumerate}
    \item \textbf{Verificación de la Segunda Ley de Newton}: Mantener masa constante mientras se varía la fuerza, observando la relación lineal con la aceleración
    \item \textbf{Experimentos de Energía}: Configurar diferentes masas y velocidades para analizar cómo afectan la energía cinética calculada
    \item \textbf{Ley de Gravitación Universal}: Experimentar con diferentes combinaciones de masas y distancias, observando cómo la fuerza gravitacional disminuye con el cuadrado de la distancia
    \item \textbf{Análisis de Relaciones}: Crear tablas de datos que muestren las relaciones proporcionales e inversamente proporcionales entre variables
\end{enumerate}

\subsubsection{Simulador de Estática}
\begin{enumerate}
    \item \textbf{Configuración del Sistema}: Comprender la disposición física de las dos esferas entre superficies en L y el papel de la cuerda tensora
    \item \textbf{Análisis de Equilibrio}: Modificar masas y observar cómo el sistema ajusta automáticamente el ángulo y la tensión para mantener el equilibrio
    \item \textbf{Estudio de Diagramas de Cuerpo Libre}: Comparar los diagramas de ambas esferas, identificando todas las fuerzas actuantes y sus direcciones
    \item \textbf{Experimentos con Fricción}: Variar los coeficientes de rozamiento y observar su efecto en la estabilidad del sistema y las fuerzas requeridas
\end{enumerate}

\subsection{Actividades de Experimentación Sistemática}

Para cada simulador, se recomienda seguir protocolos experimentales estructurados:

\subsubsection{Protocolo para Cinemática}
\begin{itemize}
    \item \textbf{Variación Controlada}: Cambiar un parámetro a la vez (ej: ángulo en tiro parabólico) manteniendo constantes los demás
    \item \textbf{Medición de Efectos}: Registrar cómo cada cambio afecta alcance, altura máxima y tiempo de vuelo
    \item \textbf{Comparación Teórico-Práctica}: Para condiciones ideales, verificar que los resultados coincidan con las ecuaciones cinemáticas
\end{itemize}

\subsubsection{Protocolo para Dinámica}
\begin{itemize}
    \item \textbf{Verificación de Proporcionalidades}: Demostrar relaciones F-m-a mediante experimentos controlados
    \item \textbf{Análisis Cuantitativo}: Calcular valores esperados y comparar con resultados del simulador
    \item \textbf{Exploración de Límites}: Probar valores extremos de parámetros para identificar comportamientos no lineales
\end{itemize}

\subsubsection{Protocolo para Estática}
\begin{itemize}
    \item \textbf{Pruebas de Estabilidad}: Determinar combinaciones de parámetros que llevan al sistema al límite del equilibrio
    \item \textbf{Análisis de Fuerzas Críticas}: Identificar qué fuerzas se vuelven determinantes en diferentes configuraciones
    \item \textbf{Optimización del Sistema}: Encontrar configuraciones que minimicen la tensión requerida o maximicen la estabilidad
\end{itemize}

\subsection{Recolección y Análisis de Datos}

Durante todas las actividades, es fundamental implementar prácticas sistemáticas de recolección y análisis:

\begin{enumerate}
    \item \textbf{Registro Metódico}: Documentar todas las configuraciones utilizadas y los resultados observados
    \item \textbf{Captura de Evidencia}: Tomar capturas de pantalla de configuraciones interesantes y gráficas significativas
    \item \textbf{Análisis Comparativo}: Crear tablas que comparen múltiples ejecuciones con variaciones controladas
    \item \textbf{Identificación de Patrones}: Buscar relaciones matemáticas y comportamientos recurrentes en los datos
\end{enumerate}

\subsection{Actividades de Síntesis y Aplicación}

Para consolidar el aprendizaje, se proponen actividades de aplicación práctica:

\begin{itemize}
    \item \textbf{Resolución de Problemas}: Utilizar los simuladores para verificar soluciones a problemas de física tradicionales
    \item \textbf{Diseño de Experimentos}: Plantear hipótesis y diseñar experimentos virtuales para verificarlas
    \item \textbf{Análisis de Casos Reales}: Modelar situaciones del mundo real (deportes, ingeniería, naturaleza) usando los simuladores
    \item \textbf{Presentación de Hallazgos}: Organizar los resultados en informes estructurados que incluyan gráficas, tablas y conclusiones
\end{itemize}

Esta metodología integral asegura que los usuarios desarrollen no solo comprensión conceptual sino también habilidades prácticas de investigación científica, análisis de datos y pensamiento crítico, todo dentro de un entorno de aprendizaje interactivo y accesible desde cualquier dispositivo con navegador web.